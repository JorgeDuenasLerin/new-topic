\documentclass{upm-template}
\usepackage{graphicx}
\usepackage{multicol}
\usepackage{enumitem}
\usepackage{tcolorbox}

\usepackage[svgnames]{xcolor}
\usepackage[tikz]{bclogo}



\newcommand{\rellena}[2]{%
  \begingroup
  \edef\temp{\noexpand\begin{tcolorbox}[
    colframe=black!80!white,
    colback=white,
    title={#1},
    height=#2cm]}%
  \temp
  \end{tcolorbox}
  \endgroup
}



\newenvironment{ojonota}
{\begin{bclogo}[logo=\bcattention, couleurBarre=red, noborder=true, 
couleur=LightGoldenrodYellow]{Nota!}}
{\end{bclogo}}


\title{Título de la Tarea}
\author{Nombre del Autor}
\date{Fecha}

\begin{document}

\maketitle
\tableofcontents
\newpage

\section{Introducción}

Este es un documento de ejemplo que muestra cómo usar la plantilla de tareas. Aquí se incluye una introducción al tema que se va a tratar en la tarea.

\subsection{Subsección de ejemplo}

Este es un texto de ejemplo para una subsección. Aquí puedes incluir información adicional sobre el tema.

\subsubsection{Sub-subsección}

Y este es un tercer nivel de profundidad en la estructura del documento.

\section{Ejemplo de Listas}

\subsection{Lista con viñetas}

Ejemplo de una lista no numerada:

\begin{itemize}
    \item Primer elemento de la lista
    \item Segundo elemento de la lista
    \item Tercer elemento de la lista
    \item \textbf{Elemento en negrita}
    \item \textit{Elemento en cursiva}
\end{itemize}

\subsection{Lista numerada}

Ejemplo de una lista numerada:

\begin{enumerate}
    \item Primer paso
    \item Segundo paso
    \item Tercer paso
\end{enumerate}

\section{Ejemplo de Código}

Aquí hay un ejemplo de cómo incluir código o comandos:

\begin{verbatim}
    # Este es un ejemplo de código
    echo "Hola Mundo"
    cd /directorio/ejemplo
    ls -la
\end{verbatim}

\section{Ejemplo de Mensajes Especiales}

\subsection{Mensaje de Nota}

\begin{ojonota}
    Este es un mensaje de nota importante. Utiliza este formato para destacar información relevante que el estudiante debe tener en cuenta.
    
    Puedes incluir varios párrafos dentro de la nota.
\end{ojonota}

\subsection{Mensaje de Atención con bclogo}

\begin{bclogo}[logo=\bcattention, couleurBarre=red, noborder=true, 
    couleur=LightGoldenrodYellow]{Atención!}
    Este es un mensaje de atención con fondo amarillo. Úsalo para advertencias o información crítica.
    
    Es ideal para destacar requisitos importantes o pasos que no se deben omitir.
\end{bclogo}

\subsection{Otro ejemplo de bclogo}

\begin{bclogo}[logo=\bcinfo, couleurBarre=blue, noborder=true, 
    couleur=LightCyan]{Información}
    Este es un mensaje informativo con otro estilo. Puedes personalizar los colores y el icono según tus necesidades.
\end{bclogo}

\section{Espacio para Rellenar}

Este es un ejemplo de cómo crear espacios para que los estudiantes completen:

\rellena{¿Cuál es la respuesta a esta pregunta?}{3}

\rellena{Describe aquí el concepto principal}{4}

\section{Ejemplo de Tablas}

\begin{center}
\begin{tabular}{|l|c|r|}
\hline
\textbf{Columna 1} & \textbf{Columna 2} & \textbf{Columna 3} \\
\hline
Dato 1 & Dato 2 & Dato 3 \\
Dato 4 & Dato 5 & Dato 6 \\
Dato 7 & Dato 8 & Dato 9 \\
\hline
\end{tabular}
\end{center}

\section{Ejercicios}

\subsection{Ejercicio 1}

Describe aquí el primer ejercicio que los estudiantes deben completar.

\rellena{Espacio para la respuesta del ejercicio 1}{5}

\subsection{Ejercicio 2}

Describe aquí el segundo ejercicio.

\begin{verbatim}
    # Ejemplo de código para el ejercicio
    comando_ejemplo --opcion valor
\end{verbatim}

\rellena{Espacio para la respuesta del ejercicio 2}{4}

\section{Conclusión}

Aquí se incluye un resumen de lo aprendido en la tarea:

\begin{enumerate}
    \item Primer concepto clave
    \item Segundo concepto clave
    \item Tercer concepto clave
    \item Cuarto concepto clave
\end{enumerate}

Este documento proporciona una base para crear tareas estructuradas y profesionales.

\end{document}